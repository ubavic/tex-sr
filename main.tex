\documentclass[11pt]{article}

\usepackage{fontspec}
\setmainfont{CMU Serif}

\usepackage{polyglossia}
\setmainlanguage{serbian}

\usepackage{xcolor}

\usepackage{graphicx}
\graphicspath{{img/}}

\usepackage{amsmath}
\usepackage{amssymb}
\usepackage{amsthm}
\theoremstyle{definition}
\newtheorem{teorema}{Теорема}
\newtheorem{primer}{Пример}

\usepackage{hyperref}
\hypersetup{
  colorlinks,
  linkcolor={red!40!black},
  citecolor={green!40!black},
  urlcolor={blue!40!black}
}

\author{Петар Петровић}
\title{Панграм}
\date{1. јануар 2022}

\begin{document}

\maketitle

Панграм (грч. πᾶν γράμμα) је реченица у којој се користе сва слова писма бар једном. 

\emph{Брза вижљаста лија хоће да ђипи преко њушке флегматичног џукца.}
Brza vižljasta lija hoće da đipi preko njuške flegmatičnog džukca.
The quick brown fox jumps over the lazy dog
A rápida raposa castanha salta por cima do cão lento.
Шустрая бурая лисица прыгает через ленивого пса.

\begin{teorema}
Лија је ђипила преко њушке џукца.
\end{teorema}

\end{document}  